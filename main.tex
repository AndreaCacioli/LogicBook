\documentclass[
	11pt, % Default font size, select one of 10pt, 11pt or 12pt
	fleqn, % Left align equations
	a4paper, % Paper size, use either 'a4paper' for A4 size or 'letterpaper' for US letter size
	%oneside, % Uncomment for oneside mode, this doesn't start new chapters and parts on odd pages (adding an empty page if required), this mode is more suitable if the book is to be read on a screen instead of printed
]{LegrandOrangeBook}

% Book information for PDF metadata, remove/comment this block if not required 
\hypersetup{
	pdftitle={Logic for Computer Science: A Practical Approach}, % Title field
	pdfauthor={Andrea Cacioli}, % Author field
	pdfsubject={Logic, Mathematics}, % Subject field
	pdfkeywords={Logic, Computer Science, Mathematics, Prolog}, % Keywords
	pdfcreator={Andrea Cacioli}, % Content creator field
}

\addbibresource{sample.bib} % Bibliography file

\definecolor{ocre}{RGB}{243, 102, 25} % Define the color used for highlighting throughout the book

\chapterimage{orange1.jpg} % Chapter heading image
\chapterspaceabove{6.5cm} % Default whitespace from the top of the page to the chapter title on chapter pages
\chapterspacebelow{6.75cm} % Default amount of vertical whitespace from the top margin to the start of the text on chapter pages

%----------------------------------------------------------------------------------------

\begin{document}

%----------------------------------------------------------------------------------------
%	TITLE PAGE
%----------------------------------------------------------------------------------------

\titlepage % Output the title page
	{\includegraphics[width=\paperwidth]{background.pdf}} % Code to output the background image, which should be the same dimensions as the paper to fill the page entirely; leave empty for no background image
	{ % Title(s) and author(s)
		\centering\sffamily % Font styling
		{\Huge\bfseries Logic for Computer Science\par} % Book title
		\vspace{16pt} % Vertical whitespace
		{\LARGE A Practical Approach\par} % Subtitle
		\vspace{24pt} % Vertical whitespace
		{\huge\bfseries Andrea Cacioli\par} % Author name
	}

%----------------------------------------------------------------------------------------
%	COPYRIGHT PAGE
%----------------------------------------------------------------------------------------

\thispagestyle{empty} % Suppress headers and footers on this page

~\vfill % Push the text down to the bottom of the page

\noindent Copyright \copyright\ 2022 Andrea Cacioli\\ % Copyright notice

\noindent \textsc{Published by Nobody probably}\\ % Publisher

\noindent \textsc{\href{https://andreacacioli.netlify.app}{andreacacioli.netlify.app}}\\ % URL

\noindent Licensed under the Creative Commons Attribution-NonCommercial 4.0 License (the ``License''). You may not use this file except in compliance with the License. You may obtain a copy of the License at \url{https://creativecommons.org/licenses/by-nc-sa/4.0}. Unless required by applicable law or agreed to in writing, software distributed under the License is distributed on an \textsc{``as is'' basis, without warranties or conditions of any kind}, either express or implied. See the License for the specific language governing permissions and limitations under the License.\\ % License information, replace this with your own license (if any)

\noindent \textit{First printing, Probably Never} % Printing/edition date

%----------------------------------------------------------------------------------------
%	TABLE OF CONTENTS
%----------------------------------------------------------------------------------------

\pagestyle{empty} % Disable headers and footers for the following pages

\tableofcontents % Output the table of contents

\listoffigures % Output the list of figures, comment or remove this command if not required

\listoftables % Output the list of tables, comment or remove this command if not required

\pagestyle{fancy} % Enable default headers and footers again

\cleardoublepage % Start the following content on a new page


\part{Introduction To Logic}

\chapterimage{orange2.jpg} % Chapter heading image
\chapterspaceabove{6.75cm} % Whitespace from the top of the page to the chapter title on chapter pages
\chapterspacebelow{7.25cm} % Amount of vertical whitespace from the top margin to the start of the text on chapter pages

%------------------------------------------------

\chapter{Logic as a field of study}

\section{Motivation}
I am a human. As such I tend to reason following what we call logic.
Logic is the foundamental layer that is behind human reasoning. Some say it probably is the most foundamental feature that makes us humans and that it actually defines us more than how much writing does. After all, it is reasonable that people started thinking before they started writing.
Studying the logic is much like studying computer science as in both activities you explore the way you would solve a problem yourself and then you try to extract the basic steps that lead you to the solution.
Unlike when writing a program though, in logic we are not faced with the dreadful experience of dealing with a compiler, which is the most pedantic and obnoxious thing about software as much as it is the most useful tool. For this reason, I would argue that logic, even though it is considered to be incredibly hard and formal, it is in fact easier than writing programs as long as you manage to really uncover the internal processes of your brain in the most basic way possible, just like when you try to write an algorithm of any sort.
With these premises, we are ready to adventure on a journey that will formalise our way of thinking and it will probably stick to your mind so much that you will feel like you could never do without it. 


\part{Propositional Logic}
\chapterimage{orange2.jpg} % Chapter heading image
\chapterspaceabove{6.75cm} % Whitespace from the top of the page to the chapter title on chapter pages
\chapterspacebelow{7.25cm} % Amount of vertical whitespace from the top margin to the start of the text on chapter pages

%------------------------------------------------

\chapter{Introduction}
\section{History}
Propositional logic is a type of logic that was first used by Crysippus in ancient Greece, where it was used to first speak of logic as a subject.
It was then brought on by the Stoics.
It was different from the Aristotelian logic, also known as syllogistic logic, that instead relied heavily on the use of terms which we do not explore in this chapter.
The original propotitional logic though did not survive the passing of time as the first books were already lost in the late antiquity. 
It was only in medieval times that, thanks to the french philosofer Peter Alebard that, we were able to basically redefine all the formalisms that were in used by the stoics.
The final refinement was made by Leibniz and later by Boole, who you might know if you worked with any programming language, and De Morgan.

\section{Features}
Propositional Logic, as the name suggests, makes strong use of something called propositions. 
A proposition is a statement regarding something that can either be true or false.
It is the relation between propositions that really is the subject of study.

Let's take these two propositions as an example.

$$
A: \text{The sky is blue} 
\newline 
B: \text{The number one seed NBA team is the Celtics}
$$




\stopcontents[part] % Manually stop the 'part' table of contents here so the previous Part page table of contents doesn't list the following chapters

%----------------------------------------------------------------------------------------
%	BIBLIOGRAPHY
%----------------------------------------------------------------------------------------

\chapterimage{} % Chapter heading image
\chapterspaceabove{2.5cm} % Whitespace from the top of the page to the chapter title on chapter pages
\chapterspacebelow{2cm} % Amount of vertical whitespace from the top margin to the start of the text on chapter pages

%------------------------------------------------

\chapter*{Bibliography}
\markboth{\sffamily\normalsize\bfseries Bibliography}{\sffamily\normalsize\bfseries Bibliography} % Set the page headers to display a Bibliography chapter name
\addcontentsline{toc}{chapter}{\textcolor{ocre}{Bibliography}} % Add a Bibliography heading to the table of contents

\section*{Articles}
\addcontentsline{toc}{section}{Articles} % Add the Articles subheading to the table of contents

\printbibliography[heading=bibempty, type=article] % Output article bibliography entries

\section*{Books}
\addcontentsline{toc}{section}{Books} % Add the Books subheading to the table of contents

\printbibliography[heading=bibempty, type=book] % Output book bibliography entries

%----------------------------------------------------------------------------------------
%	INDEX
%----------------------------------------------------------------------------------------

\cleardoublepage % Make sure the index starts on an odd (right side) page
\phantomsection
\addcontentsline{toc}{chapter}{\textcolor{ocre}{Index}} % Add an Index heading to the table of contents
\printindex % Output the index

%----------------------------------------------------------------------------------------
%	APPENDICES
%----------------------------------------------------------------------------------------

\chapterimage{orange2.jpg} % Chapter heading image
\chapterspaceabove{6.75cm} % Whitespace from the top of the page to the chapter title on chapter pages
\chapterspacebelow{7.25cm} % Amount of vertical whitespace from the top margin to the start of the text on chapter pages

\begin{appendices}

\renewcommand{\chaptername}{Appendix} % Change the chapter name to Appendix, i.e. "Appendix A: Title", instead of "Chapter A: Title" in the headers

%------------------------------------------------

\chapter{Appendix Chapter Title}

\section{Appendix Section Title}

Lorem ipsum dolor sit amet, consectetur adipiscing elit. Aliquam auctor mi risus, quis tempor libero hendrerit at. Duis hendrerit placerat quam et semper. Nam ultricies metus vehicula arcu viverra, vel ullamcorper justo elementum. Pellentesque vel mi ac lectus cursus posuere et nec ex. Fusce quis mauris egestas lacus commodo venenatis. Ut at arcu lectus. Donec et urna nunc. Morbi eu nisl cursus sapien eleifend tincidunt quis quis est. Donec ut orci ex. Praesent ligula enim, ullamcorper non lorem a, ultrices volutpat dolor. Nullam at imperdiet urna. Pellentesque nec velit eget est euismod pretium.

%------------------------------------------------

\chapter{Appendix Chapter Title}

\section{Appendix Section Title}

Lorem ipsum dolor sit amet, consectetur adipiscing elit. Aliquam auctor mi risus, quis tempor libero hendrerit at. Duis hendrerit placerat quam et semper. Nam ultricies metus vehicula arcu viverra, vel ullamcorper justo elementum. Pellentesque vel mi ac lectus cursus posuere et nec ex. Fusce quis mauris egestas lacus commodo venenatis. Ut at arcu lectus. Donec et urna nunc. Morbi eu nisl cursus sapien eleifend tincidunt quis quis est. Donec ut orci ex. Praesent ligula enim, ullamcorper non lorem a, ultrices volutpat dolor. Nullam at imperdiet urna. Pellentesque nec velit eget est euismod pretium.

%------------------------------------------------

\end{appendices}

%----------------------------------------------------------------------------------------

\end{document}
